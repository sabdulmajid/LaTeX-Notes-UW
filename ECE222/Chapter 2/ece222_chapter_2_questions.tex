\documentclass[11pt]{article}
\usepackage[utf8]{inputenc}
\usepackage[margin=0.57in]{geometry}

\begin{document}

\section*{Chapter 2 Questions - Assignment Questions for Week 2}

\begin{enumerate}
    \item[\textbf{2.1}] \textbf{<§2.2>} For the following C statement, write the corresponding RISC-V 
    assembly code. Assume that the C variables \texttt{f}, \texttt{g}, and \texttt{h}, have already been placed 
    in registers \texttt{x5}, \texttt{x6}, and \texttt{x7} respectively. Use a minimal number of RISC-V assembly 
    instructions.
    \begin{verbatim}
    f = g + (h - 5);
    \end{verbatim}

    \item[\textbf{2.2}] \textbf{<§2.2>} Write a single C statement that corresponds to the two RISC-V 
    assembly instructions below.
    \begin{verbatim}
    add f, g, h
    add f, i, f
    \end{verbatim}

    \item[\textbf{2.3}] \textbf{<§§2.2, 2.3>} For the following C statement, write the corresponding 
    RISC-V assembly code. Assume that the variables \texttt{f}, \texttt{g}, \texttt{h}, \texttt{i}, and \texttt{j} are assigned to 
    registers \texttt{x5}, \texttt{x6}, \texttt{x7}, \texttt{x28}, and \texttt{x29}, respectively. Assume that the base address 
    of the arrays \texttt{A} and \texttt{B} are in registers \texttt{x10} and \texttt{x11}, respectively.
    \begin{verbatim}
    B[8] = A[i - j];
    \end{verbatim}

    \item[\textbf{2.4}] \textbf{<§§2.2, 2.3>} For the RISC-V assembly instructions below, what is the 
    corresponding C statement? Assume that the variables \texttt{f}, \texttt{g}, \texttt{h}, \texttt{i}, and \texttt{j} are assigned 
    to registers \texttt{x5}, \texttt{x6}, \texttt{x7}, \texttt{x28}, and \texttt{x29}, respectively. Assume that the base 
    address of the arrays \texttt{A} and \texttt{B} are in registers \texttt{x10} and \texttt{x11}, respectively.
    \begin{verbatim}
    slli x30, x5, 3 // x30 = f*8
    add x30, x10, x30 // x30 = &A[f]
    slli x31, x6, 3 // x31 = g*8
    add x31, x11, x31 // x31 = &B[g]
    ld x5, 0(x30) // f = A[f]
    addi x12, x30, 8
    ld x30, 0(x12)
    add x30, x30, x5
    sd x30, 0(x31)
    \end{verbatim}

    \item[\textbf{2.5}] \textbf{<§2.3>} Show how the value \texttt{0xabcdef12} would be arranged in memory 
    of a little-endian and a big-endian machine. Assume the data is stored starting at 
    address 0 and that the word size is 4 bytes.

    \item[\textbf{2.7}] \textbf{<§§2.2, 2.3>} Translate the following C code to RISC-V. Assume that the 
    variables \texttt{f}, \texttt{g}, \texttt{h}, \texttt{i}, and \texttt{j} are assigned to registers \texttt{x5}, \texttt{x6}, \texttt{x7}, \texttt{x28}, and \texttt{x29}, 
    respectively. Assume that the base address of the arrays \texttt{A} and \texttt{B} are in registers \texttt{x10}
    and \texttt{x11}, respectively. Assume that the elements of the arrays \texttt{A} and \texttt{B} are 8-byte 
    words:
    \begin{verbatim}
    B[8] = A[i] + A[j];
    \end{verbatim}

    \item[\textbf{2.10}] Assume that registers \texttt{x5} and \texttt{x6} hold the values \texttt{0x8000000000000000}
    and \texttt{0xD000000000000000}, respectively.
\end{enumerate}

\end{document}
