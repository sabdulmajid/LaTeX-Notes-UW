\documentclass[11pt]{article}
\usepackage[utf8]{inputenc}
\usepackage[margin=0.57in]{geometry}

\begin{document}

\section*{Chapter 2 Questions - Assignment Questions for Week 2}

\subsection*{2.1}
\begin{verbatim}
[2.1] [5] <§2.2> For the following C statement, write the corresponding RISC-V 
assembly code. Assume that the C variables f, g, and h, have already been placed 
in registers x5, x6, and x7 respectively. Use a minimal number of RISC-V assembly 
instructions.
f = g + (h - 5);
\end{verbatim}

\subsection*{2.2}
\begin{verbatim}
[2.2] [5] <§2.2> Write a single C statement that corresponds to the two RISC-V 
assembly instructions below.
add f, g, h
add f, i, f
\end{verbatim}

\subsection*{2.3}
\begin{verbatim}
[2.3] [5] <§§2.2, 2.3> For the following C statement, write the corresponding 
RISC-V assembly code. Assume that the variables f, g, h, i, and j are assigned to 
registers x5, x6, x7, x28, and x29, respectively. Assume that the base address 
of the arrays A and B are in registers x10 and x11, respectively.
B[8] = A[i - j];
\end{verbatim}

\subsection*{2.4}
\begin{verbatim}
[2.4] [10] <§§2.2, 2.3> For the RISC-V assembly instructions below, what is the 
corresponding C statement? Assume that the variables f, g, h, i, and j are assigned 
to registers x5, x6, x7, x28, and x29, respectively. Assume that the base 
address of the arrays A and B are in registers x10 and x11, respectively.

slli x30, x5, 3 // x30 = f*8
add x30, x10, x30 // x30 = &A[f]
slli x31, x6, 3 // x31 = g*8
add x31, x11, x31 // x31 = &B[g]
ld x5, 0(x30) // f = A[f]
addi x12, x30, 8
ld x30, 0(x12)
add x30, x30, x5
sd x30, 0(x31)
\end{verbatim}

\subsection*{2.5}
\begin{verbatim}
[2.5] [5] <§2.3> Show how the value 0xabcdef12 would be arranged in memory 
of a little-endian and a big-endian machine. Assume the data are stored starting at 
address 0 and that the word size is 4 bytes.
\end{verbatim}

\subsection*{2.7}
\begin{verbatim}
[2.7] [5] <§§2.2, 2.3> Translate the following C code to RISC-V. Assume that the 
variables f, g, h, i, and j are assigned to registers x5, x6, x7, x28, and x29, 
respectively. Assume that the base address of the arrays A and B are in registers x10
and x11, respectively. Assume that the elements of the arrays A and B are 8-byte 
words:
B[8] = A[i] + A[j];
\end{verbatim}

\subsection*{2.10}
\begin{verbatim}
Assume that registers x5 and x6 hold the values 0x8000000000000000
and 0xD000000000000000, respectively.
\end{verbatim}

\end{document}
